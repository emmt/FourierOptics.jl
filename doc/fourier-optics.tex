% !TeX spellcheck = en_US
\documentclass[a4paper,10pt,twocolumn]{scrartcl}

%\renewcommand{\baselinestretch}{1.0} % Change to 1.65 for double spacing
%\hoffset-2.5mm
%\voffset-1mm
%\usepackage[T1]{fontenc}
\usepackage[utf8]{inputenc}
%\usepackage{kpfonts}
%\usepackage{lmodern}
%\usepackage{libertine}
\usepackage{fourier} % obvious choice for this note ;-)
%\usepackage{bera}
%\usepackage{dejavu}

\newcommand{\Title}{Fourier Optics Simulations}
\newcommand{\Authors}{É.\ Thiébaut \& M. Tallon}

\title{\Title}
\author{\Authors}

\usepackage[left=9mm, right=9mm, top=20mm, bottom=20mm,columnsep=8mm]{geometry}

\DeclareUnicodeCharacter{2264}{\le} % ≤
\DeclareUnicodeCharacter{2265}{\ge} % ≥
\DeclareUnicodeCharacter{22C5}{\!\cdot\!}

%\newcommand*{\DeleteColor}{\color{orange}}
%\newcommand*{\NewColor}{\color{teal}}
%\newcommand{\BeginDelete}{\begingroup\DeleteColor}
%\let\EndDelete=\endgroup
%\newcommand{\BeginNew}{\begingroup\NewColor}
%\let\EndNew=\endgroup
%\newcommand{\Delete}[1]{{\DeleteColor\sout{#1}}}
%\newcommand{\New}[1]{{\NewColor#1}}
%\newcommand{\Replace}[2]{\Delete{#1} \New{#2}}
\newcommand{\Hide}[1]{}
\newcommand{\Show}[1]{#1}

%\usepackage{bbm}
\usepackage{dsfont}
\RequirePackage{mathrsfs}
%\RequirePackage{stmaryrd}

%\usepackage{subfig}

\usepackage{graphicx}
\graphicspath{{cholesky-figs/}}

\usepackage{amsmath,amsfonts,amssymb}
\usepackage{amsthm}
\usepackage{mathtools}
%\usepackage[authoryear,round,semicolon]{natbib}

\usepackage[squaren,Gray]{SIunits}
\usepackage{xspace}

\usepackage[ruled,vlined]{algorithm2e}
\usepackage{algpseudocode}

%\usepackage{tikz}
%\usetikzlibrary{calc}
%\usetikzlibrary{arrows.meta}

% Table settings:
%\usepackage{array}
%\usepackage{booktabs}
%\colorlet{tableheadcolor}{gray!25}
%\colorlet{tablerowcolor}{gray!12.5}
%\newcommand{\TopRule}{\specialrule{\heavyrulewidth}{0pt}{0pt}}
%\newcommand{\BottomRule}{\specialrule{\lightrulewidth}{0pt}{0pt}}
%\extrarowheight2pt
%\newcolumntype{P}[1]{>{\raggedright}p{#1}}
%\newcolumntype{M}[1]{>{\raggedright}m{#1}}


\usepackage[dvipsnames,hyperref]{xcolor}
\newcommand{\oops}[1]{\textcolor[named]{BrickRed}{#1}}
\colorlet{HyperColor}{MidnightBlue}
\colorlet{ChapColor}{OliveGreen}
\colorlet{MyLinkColor}{HyperColor}
\colorlet{MyURLColor}{HyperColor}
\colorlet{MyCiteColor}{HyperColor}
\colorlet{MyFileColor}{HyperColor}
\colorlet{MyMenuColor}{HyperColor}
\colorlet{MyPageColor}{HyperColor}

\usepackage[unicode=true,
 bookmarks=true,bookmarksnumbered=false,bookmarksopen=false,
 breaklinks=true,pdfborder={0 0 0},backref=page,colorlinks=true]{hyperref}
\hypersetup{
  pdftitle={\Title},
  pdfauthor={\Authors},
  unicode=true,
  linkcolor=MyLinkColor,
  urlcolor=MyURLColor,
  citecolor=MyCiteColor,
  filecolor=MyFileColor,
  menucolor=MyMenuColor%,pagecolor=MyPageColor
}

\newcommand{\newstuff}[1]{{\color{magenta}#1}}
\newcommand{\strong}[1]{\textbf{\emph{#1}}}
\newcommand*{\code}[1]{\texttt{#1}}
\newcommand*{\variable}[1]{\texttt{#1}}
\newcommand*{\FileName}[1]{\texttt{#1}}

% Abbreviations:
\newcommand*{\etal}{\emph{et al.}\xspace}
\newcommand*{\etc}{\emph{etc.}\xspace}
\newcommand*{\eg}{\emph{e.g.}\xspace}
\newcommand*{\ie}{\emph{i.e.}\xspace}

%%%%%%%%%%%%%%%%%%%%%%%%%%%%%%%%%%%%%%%%%%%%%%%%%%%%%%%%%%%%%%%%%%%%%%%%%%%%%%%%
%\useosf % no longer required if osf specified, otherwise after all math
\DeclareSymbolFont{bbold}{U}{bbold}{m}{n}
\DeclareSymbolFontAlphabet{\mathbbold}{bbold}
\RequirePackage{textcomp}  % required for special glyphs

% Parenthesis:
%\def\Paren{}\let\Paren\undefined
\DeclarePairedDelimiterX{\Paren}[1]{(}{)}{#1}
\DeclarePairedDelimiterX{\Brace}[1]{\{}{\}}{#1}
\DeclarePairedDelimiterX{\Brack}[1]{[}{]}{#1}
\DeclarePairedDelimiterX{\Abs}[1]{\rvert}{\lvert}{#1}
\DeclarePairedDelimiterX{\Norm}[1]{\lVert}{\rVert}{#1}
\DeclarePairedDelimiterX{\Group}[1]{\lgroup}{\rgroup}{#1}
%\DeclarePairedDelimiterX{\FrobNorm}[1]{\lVert}{\rVert_{\Tag{F}}}{#1}
\DeclarePairedDelimiterX{\Avg}[1]{\langle}{\rangle}{#1}
\DeclarePairedDelimiterX{\Round}[1]{\lfloor}{\rceil}{#1}
\DeclarePairedDelimiterX{\Floor}[1]{\lfloor}{\rfloor}{#1}
\DeclarePairedDelimiterX{\Ceil}[1]{\lceil}{\rceil}{#1}
%\DeclarePairedDelimiterX{\Inner}[2]{\langle}{\rangle}{#1\,\delimsize\vert\,#2}
\DeclarePairedDelimiterX{\Inner}[2]{\langle}{\rangle}{#1,#2}
%\DeclarePairedDelimiterXPP{\IndexedList}[3]{}\{\}{_{#2}^{#3}}{#1}
\DeclarePairedDelimiterXPP{\IndexedList}[3]{}\{\}{_{#2,\ldots,#3}}{#1}

% A list \List{ELEM}{START}{END}
\DeclarePairedDelimiterXPP{\List}[3]{}{\{}{\}}{_{#2,\ldots,#3}}{#1}

\newcommand*{\delimsize}{}
\newcommand*{\suchthat}{\,\vert\,} % compact version
\newcommand*{\SuchThat}{\:\delimsize\vert\:} % autoscale to surrounding version
\newcommand*{\given}{\,\vert\,} % compact version
\newcommand*{\Given}{\:\delimsize\vert\:} % autoscale to surrounding version
%\newcommand*{\given}{\mathbin{\vert}}
%\newcommand*{\Given}{\mathrel{\delimsize\vert}} % autoscale to surrounding version

% Upright letters:
\newcommand*{\mathd}{\mathrm{d}}
\newcommand*{\mathe}{\mathrm{e}}
\newcommand*{\mathi}{\mathrm{i}}

% Operators/functions
\renewcommand*{\Re}{\operatorname{Re}}
\renewcommand*{\Im}{\operatorname{Im}}
\DeclareMathOperator*{\argmin}{arg\,min}
\DeclareMathOperator*{\argmax}{arg\,max}
\DeclareMathOperator{\arc}{arc}
\DeclareMathOperator{\rnd}{rnd}
\DeclareMathOperator{\sign}{sign}
\DeclareMathOperator{\sinc}{sinc}
\DeclareMathOperator{\Card}{Card}
\DeclareMathOperator{\Conv}{conv}
%\DeclareMathOperator{\Det}{Det}
\newcommand*{\Det}{\det}
\DeclareMathOperator{\Diag}{diag}
\DeclareMathOperator{\Dom}{dom}
\DeclareMathOperator{\Prox}{prox}
\DeclareMathOperator{\Rank}{rank}
\DeclareMathOperator{\Sign}{sign}
\DeclareMathOperator{\Span}{span}
\DeclareMathOperator{\Trace}{tr}
\newcommand*{\trace}{\Trace}

% Statistics:
%\DeclareMathOperator{\Var}{Var}
%\DeclareMathOperator{\Cov}{Cov}
%\DeclareMathOperator{\Expect}{E}
%\newcommand*{\Expect}{\mathbb{E}}
\DeclarePairedDelimiterXPP{\Var}[1]{\mathrm{Var}}(){}{#1}
\DeclarePairedDelimiterXPP{\Cov}[1]{\mathrm{Cov}}(){}{#1}
\DeclarePairedDelimiterXPP{\Expect}[1]{\mathrm{E}}(){}{#1}
\DeclarePairedDelimiterXPP{\LogPr}[1]{\ell}(){}{#1}

% Sets:
\newcommand*{\Set}[1]{\mathbb{#1}}
\newcommand*{\Reals}{\Set{R}}
\newcommand*{\Integers}{\Set{Z}}
\newcommand*{\NaturalNumbers}{\Set{N}}
\newcommand*{\Rationals}{\Set{Q}}
\newcommand*{\Complexes}{\Set{C}}

% Miscellaneous:
\newcommand*{\Tag}[1]{\mathsf{#1}}
\newcommand*{\PDer}[2]{\frac{\partial#1}{\partial#2}}
\newcommand*{\bydef}{\stackrel{{\scriptscriptstyle \text{def}}}{=}}
\newcommand*{\TwoIPi}{2\,\mathi\,\pi}
%\newcommand*{\rand}[1]{\widetilde{#1}} % random value/variable
%\newcommand*{\rand}[1]{\tilde{#1}} % random value/variable
%\newcommand*{\SIM}{\text{\raisebox{-0.3ex}[0pt][0pt]{$\sim$}}}
\newcommand*{\SIM}{\smash{\sim}}
\newcommand*{\proxy}[1]{\overset{\SIM}{#1}} % random value/variable
%\newcommand*{\proxy}[1]{\widetilde{#1}}
\newcommand*{\estim}[1]{\widehat{#1}}
%\newcommand*{\FT}[1]{\widehat{#1}}
%\newcommand*{\FT}[1]{\mathring{#1}}
\newcommand*{\FT}[1]{\widetilde{#1}}
\newcommand*{\DFT}[1]{\widehat{#1}}
\newcommand*{\MathNote}[2]{\hspace*{#1}\text{\footnotesize(#2)}}
\newcommand*{\better}[1]{#1^{+}}
\newcommand*{\best}[1]{#1^{\star}}
\newcommand*{\Combine}{{\circ}}
\newcommand*{\EndProof}{\blacksquare}


% ========== LINEAR ALGEBRA ==========

\newcommand*{\V}[1]{\boldsymbol{#1}}   % a vector
\newcommand*{\M}[1]{\mathbf{#1}}       % an operator (a matrix)

%\newcommand*{\TransposeLetter}{\top}
%\newcommand*{\TransposeLetter}{\mathrm{t}}
\newcommand*{\TransposeLetter}{\mathsf{T}}
\newcommand*{\T}{^{\TransposeLetter}}
\newcommand*{\mT}{^{-\TransposeLetter}}
%\newcommand*{\ConjugateTransposeLetter}{\ast}
\newcommand*{\ConjugateTransposeLetter}{\mathsf{H}}
\newcommand*{\CT}{^{\ConjugateTransposeLetter}}
\newcommand*{\mCT}{^{-\ConjugateTransposeLetter}}

% The following definition is to remove surrounding spacing around ⋅
% to denote dot product.
\newcommand*{\MDot}{\mathord{\,\mathchar"2201\,}}
\newcommand*{\Tcdot}{\T\!⋅}

\newcommand*{\QuadTerm}[2]{#2\T\MDot#1\MDot#2}
\newcommand{\ArrayWithDelimiters}[4]{%
  \left #1\begin{array}{#2}#4\end{array}\right #3}
\newcommand{\Matrix}[2]{\ArrayWithDelimiters{\lgroup}{#1}{\rgroup}{#2}}
\newcommand{\Vector}[1]{\ArrayWithDelimiters{\lgroup}{c}{\rgroup}{#1}}
\newcommand{\TwoByTwoMatrix}[4]{\Matrix{cc}{#1 & #2 \\ #3 & #4 \\}}
\newcommand{\TwoByTwoSymmetricMatrix}[3]{\TwoByTwoMatrix{#1}{#3}{#3}{#2}}

%\newcommand*{\One}{1\hspace*{-0.8ex}1}
%\newcommand*{\Zero}{0\hspace*{-0.8ex}0}
%\newcommand*{\One}{\mathbbm{1}}
%\newcommand*{\Zero}{\mathbbm{0}}
\newcommand*{\One}{\mathds{1}}
\newcommand*{\Zero}{\mathds{O}}
%\newcommand*{\Zero}{\V 0}
%\newcommand*{\One}{\V 1}

%------------------------------------------------------------------------------
% Ref: Alexander R. Perlis., "A complement to \smash, \llap, and \rlap,"
%      TUGboat, Vol. 22, pp. 350-352, 2001.
%      <http://math.arizona.edu/~aprl/publications/mathclap/>
%
% For comparison, the existing overlap macros:
% \def\llap#1{\hbox to 0pt{\hss#1}}
% \def\rlap#1{\hbox to 0pt{#1\hss}}
  \def\clap#1{\hbox to 0pt{\hss#1\hss}}
  \def\mathllap{\mathpalette\mathllapinternal}
  \def\mathrlap{\mathpalette\mathrlapinternal}
  \def\mathclap{\mathpalette\mathclapinternal}
  \def\mathllapinternal#1#2{\llap{$\mathsurround=0pt#1{#2}$}}
  \def\mathrlapinternal#1#2{\rlap{$\mathsurround=0pt#1{#2}$}}
  \def\mathclapinternal#1#2{\clap{$\mathsurround=0pt#1{#2}$}}

%----------------------------------------------------------- SIMPLE ALGORITHMS -

\usepackage{array}
\newcommand{\CodeBlock}[1]{\begin{array}{{>{\displaystyle}l}}#1\end{array}} 
\newcommand{\FloorBlock}[1]{\left\lfloor\quad\CodeBlock{#1}\right.} 
\newcommand{\VertBlock}[1]{\left\lvert\quad\CodeBlock{#1}\right.} 

%------------------------------------------------------------------- NOTATIONS -

\newcommand*{\textfrac}[2]{{\textstyle\frac{#1}{#2}}}
\newcommand*{\half}{\textfrac{1}{2}}
\newcommand*{\IntegerRange}[1]{[\hspace*{-0.4ex}[#1]\hspace*{-0.4ex}]}

\begin{document}
\maketitle

\begin{abstract}
Abstract...
\end{abstract}


\section{Numerical Simulation}

Propagation of the complex amplitude between the pupil plane and the focal
plane of a (perfect) lens amounts to computing:
\begin{align}
  \FT{A}(\V s) = \iint A(\V r) \, \mathe^{
    \frac{-2\,\mathi\,\pi}{\lambda\,f}\,\Inner{\V r}{\V s}
  } \, \mathd^2\V r \, ,
  \label{eq:lens-transform}
\end{align}
which involves a continuous Fourier transform and where $\lambda$ and $f$
are the wavelength and the focal length while $\V r$ and $\V s$ are the
positions in the pupil and focal plane respectively.  \oops{[A
normalization term is missing.]}

When the model is discretized, the complex amplitudes $A(\V r)$ and
$\FT{A}(\V s)$ are sampled at evenly spaced positions.  For instance:
\begin{subequations}
\begin{align}
  A_{\V j} &\approx A(\V r_{\V j}) \,, \\
  \FT{A}_{\V k} &\approx \FT{A}(\V s_{\V k}) \,,
\end{align}
\end{subequations}
with $\V j \in \Omega$ and $\V k \in \Omega$ the 2D indices of the
samples in, respectively, the input and output plane and:
\begin{subequations}
\begin{align}
  \V r_{\V j} &= (\V j - \V j_0) \times \Delta r \, , \\
  \V s_{\V k} &= (\V k - \V k_0) \times \Delta s \, ,
\end{align}
\end{subequations}
where $\Delta r$ and $\Delta s$ are the sampling sizes (assumed to be the
same along the two dimensions) in the pupil plane and in the focal plane
while $\V j_0$ and $\V k_0$ are suitable offsets (not necessarily
integers).  The set $\Omega \subset \Integers^2$ of sampling indices has
size $N_1 \times N_2$.  Using the same conventions as for the 2D discrete
Fourier transform, $\Omega$ is defined as:
\begin{equation}
  \Omega = \IntegerRange{0,N_1-1}\times\IntegerRange{0,N_2-1} \, .
\end{equation}
As a result of choosing identical sampling sizes along all dimensions, we
have to take the same number of samples in every dimension (hence
$N_1=N_2=N$). \oops{[OK but explain why.]}

For discretized complex amplitudes, the continuous Fourier transform is
approximated by a discrete Fourier transform (DFT) defined by:
\begin{equation}
  \DFT{B}_{\V k}
  \bydef \sum_{\V j \in \Omega}
  B_{\V j} \, \mathe^{-2\,\mathi\,\pi \, \Inner{\V j}{\V k}_{\Omega}} \, ,
  \label{eq:DFT}  
\end{equation}
where the inner product on $\Omega$ is defined by:
\begin{equation}
  \Inner{\V j}{\V k}_{\Omega} = \sum\nolimits_d \frac{j_d\,k_d}{N_d} \, ,
\end{equation}
where the sum is taken over the number of dimensions ($2$ for us).

Approximating the integral in Eq.~(\ref{eq:lens-transform}) by a Riemann sum yields:
\begin{equation}
  \FT{A}_{\V k} \approx \sum_{\V j \in \Omega} A_{\V j} \, \mathe^{\frac{-2\,\mathi\,\pi}{\lambda\,f}\,\Inner{\V r_{\V j}}{\V s_{\V k}}} \,
  \Delta r^2 \, .
  \label{eq:Riemann-approx}
\end{equation}
\oops{[There is also the missing normalization factor to take into account here.]}
In order to make this looks like a DFT, the following relation must hold:
\begin{displaymath}
  \Inner{\V j - \V j_0}{\V k - \V k_0}_{\Omega}
  = \frac{\Inner{\V r_{\V j}}{\V s_{\V k}}}{\lambda\,f} \, ,
  \label{eq:dimensioning}
\end{displaymath}
which implies that:
\begin{equation}
  N\,\Delta r\,\Delta s = \lambda\,f \, .
\end{equation}
This relation relates the sampling sizes in the two planes.  Note that the
quantities $\Delta r$, $\Delta s$, $\lambda$ and $f$ are all lengths.
Assuming the relation in Eq.~(\ref{eq:dimensioning}) holds, the complex
amplitude in the focal plane is approximately given by:
\begin{align}
  \FT{A}_{\V k}
  &\approx \sum_{\V j \in \Omega} A_{\V j} \,
  \mathe^{-2\,\mathi\,\pi \, \Inner{\V j - \V j_0}{\V k - \V k_0}_{\Omega}} \,
  \Delta r^2 \notag\\
  &= \Delta r^2 \,
  \mathe^{+2\,\mathi\,\pi \, \Inner{\V j_0}{\V k - \V k_0}_{\Omega}} \,
  \sum_{\V j \in \Omega}
  \mathe^{+2\,\mathi\,\pi \, \Inner{\V j}{\V k_0}_{\Omega}} \, A_{\V j} \, \mathe^{-2\,\mathi\,\pi \, \Inner{\V j}{\V k}_{\Omega}} \, ,  
\end{align}
where the sum in the last right hand side is the DFT of:
\begin{displaymath}
  \mathe^{+2\,\mathi\,\pi \, \Inner{\V j}{\V k_0}_{\Omega}} \, A_{\V j} \, .
\end{displaymath}
If the offsets $\V j_0$ and $\V k_0$ have integer coordinates, another way
to write this is:
\begin{equation}
  \FT{A}_{\V k + \V k_0} \approx \Delta r^2 \sum_{\V j \in \Omega}
  A_{\V j + \V j_0} \,
  \mathe^{-2\,\mathi\,\pi \, \Inner{\V j}{\V k}_{\Omega}} \, ,
\end{equation}
where circular shift is used for translating the sampled arrays.  If the
\emph{origin of coordinates} is defined following the usual convention,
then the offsets $\V j_0$ and $\V k_0$ have integer coordinates and the
equation becomes:
\begin{align}
  \FT{A}_{\V k} &\approx \Delta r^2 \,
  \mathtt{fftshift}\Paren*{
    \sum_{\V j \in \Omega}
    \mathtt{ifftshift}\Paren*{\V A}_{\V j} \,
  \mathe^{-2\,\mathi\,\pi \, \Inner{\V j}{\V k}_{\Omega}}
  }_{\V k} \\
  &= \Delta r^2 \,\mathtt{fftshift}\Paren*{\mathtt{fft}\Paren*{
    \mathtt{ifftshift}\Paren*{\V A}
  }}_{\V k} \, ,
\end{align}
where $\V A$ denotes the array of sampled values of the complex amplitude
in the pupil plane.

% References
\footnotesize
\bibliography{journals-short,biblio}
\bibliographystyle{spiebib}

\end{document}
